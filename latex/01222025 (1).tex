\documentclass{assets/fieldnotes}

\title{Kono (Sierra Leone)}
\author{LING3020/5020}
\date{University of Pennsylvania, Spring 2025\\01/22/2025 Basic Vocabulary}

\setcounter{secnumdepth}{4} %enable \paragraph -- for subsubsubsections

\begin{document}

\maketitle
\tableofcontents

\jal{Colour of Julie's notes}\\
\jk{Color of Jianjing's notes}\\
\g{Color of Giang's notes}\\
\wml{Color of Wesley's notes}\\
\mb{Color of Mingyang's notes}\\
\ds{Color of Daniel's notes} \\
\jmt{Color of Jan's notes} \\
\chs{Color of Chun-Hung's notes} \\
\jf{Color of Joey's notes} \\



\section{Basics for Politeness - Daniel} 

%\exg.  example\\
%gloss \\
%`hello' \jal{best to try "good morning/afternoon/night" at same time; generally no guarantee they're different; also check the 2023 notes}

\jal{you also got "morning", "sun", include tones when you can, notice the difference in how you've transcribed 2sg with good afternoon versus thank you}

\ex. \textipa{tS}éna\\
morning \\

\ex. t\super{h}e\\
afternoon\\

\exg. g\super{w}a\\
evening/night\\

\exg. t\super{h}ee\\
sun\\

\ds{General/formal greetings:}
\exg. éen \textipa{tS}éna\\
2\ts{sg} morning\\
`good morning' (SG)

\exg. éen t\super{h}e\\
2\ts{sg} afternoon\\
`good afternoon' (SG)

\exg. éen g\super{w}a\\
2\ts{sg} afternoon\\
`good evening/night' (SG)

\exg. wen \textipa{tS}éna\\
2\ts{pl} morning\\
`good morning' (PL)

\exg. wen t\super{h}e\\
2\ts{pl} afternoon\\
`good afternoon' (PL)

\exg. wen g\super{w}a\\
2\ts{pl} afternoon\\
`good evening/night' (PL)


\exg. m-bo éen t\super{h}e\\
1\Sg{}.\Poss{}-friend 2\Sg{}(?) day\\
`good day' (SG, informal) \ds{Commented as `short'; find out what \textit{een} really is}

\exg. mb\super{w}\textipa{E} éen t\super{h}e\\
1\Sg{}.\Poss{}-friend 2\Sg{} day\\
`good day' (SG, informal)

\exg. mb\super{w}\textipa{E}nu ween t\super{h}e\\
1\Sg{}.\Poss{}-friend 2\Sg{} day\\
`good day' (PL, informal)

\exg. mb\super{w}\textipa{E}nu ween \textipa{tS}éna\\
1\Sg{}.\Poss{}-friend 2\Pl{} morning\\
`good morning' (PL, informal)

\exg. fa éen t\super{h}e\\
father 2\Sg{} day\\
`good day' (respectful to father)

\exg. nde éen t\super{h}e\\
mother 2\Sg{} day\\
`good day' (respectful to mother)

\exg. nde éen g\super{w}a\\
mother 2\Sg{} night\\
`good night' (respectful to mother)

\exg. nde éen \textipa{tS}éna\\
mother greet.2\Sg{} morning\\
`good morning' (respectful to mother)

\exg. sa éen \textipa{tS}éna / t\super{h}e / g\super{w}a\\
Sa 2\Sg{} morning {} day {} night\\
`Good morning/day/night, Sa (name)'

\exg. m-b\super{w}\textipa{E:}-ne\\
1\Sg{}.\Poss{}-friend(?)-this(??)\\
`this friend' (of mine?) \ds{Later on m- seemed to be 1SG.POSS, is that here also?}
%-------------------

\ex. mwaa gbɛ\\
`goodbye' \ds{I think he pronounced them markedly as two words}

\ex. mbe ane\\
'bye for now/see you' \ds{Commented that both are relatively interchangeable}\\

\exg. ɛ̃ŋ ŋgwai\\
2\ts{sg} thank\\
`thank you' (SG)

\exg. wɛŋ ŋgwai\\
2\ts{pl} thank\\
`thank you' (PL)

\exg. ama jao\\
not(?) bad\\
`not bad' (response to thank you?)

\ex. jao\\
bad\\

\exg. ii jao\\
2SG bad\\

\exg. wo jao\\
2PL bad\\
\ds{Did not entirely catch what exactly these two are for}



\section{People, Kinship - Chun-Hung}

people:

\ex. mok\super{h}ama \\
`man' \chs{(originally from moɛ-k\super{h}aima `person-man')}

\ex. musu \\
woman 

\ex. k\super{h}aɛnɛ \\
`boy' \chs{(k\super{h}aɛ-diɛ `man-child' for a longer version, more articulate?) }

\ex. de-musu \\
child-female \\
`girl'

\ex. bago \\
adult; elder

\ex. tʃɛnanɛ \\
adult; elder

\ex. tʃɛna-n-k\super{h}ai \\
adult-N-man \\
adult man \chs{(but *bago-k\super{h}ai)}

\ex. tʃɛna-musu \\
adult-woman \\
adult woman \chs{(but *bago-musu)}

\ex. diɛ \\
`child'

\ex. denɛ \\
`child'

\ex. de-mɛsɛ \\
child-small \\
`baby' \chs{(de-mɛsɛ-mɛsɛ for the emphatic meaning of very small)}

\ex. denindiɛ \\
`baby' 

\ex. friend

\ex. pa-tʃɛnɛ \\
PA-house \\
`neighbor'

\ex. Kono people (how they refer to themselves)

kinship (should we include possessives when eliciting kinship terms?):\jal{only if they're required, sometimes possessors are required for inalienable possession}

\ex. fa \\
father 

\ex. dɛ \\
mother \chs{(also dɛdɛ)}



\section{Clothing - Alexis } 
\jal{worth asking about any typical clothing worn there; notice that winter/spring/summer/fall are not relevant seasons there.  Please reorder in terms of basic vocabulary first.  (e.g. "beanie" should just be deleted altogether, as should "undergarments" and "underwear", and "pants" "shirt" "shorts" and "shoes" should be much higher.)  Also think about if he doesn't know the word, will you be able to successfully explain this?  (blouse?) Note that  "wedding dress" assumes a lot of cultural background}

\exg. example\\
gloss\\
'Formal Dress' \jal{in latex, you need to use this symbol for left quotation: ` }

Q: Are there typical clothes you or the people of Sierra Leone wear there?

\ex. k၁a kæŋe \\
`weaving'

\ex. k၁a \\
`cloth'

\ex. k၁a nima \\
`nice dress'

\ex. nima \\
`nice, pretty'

\ex. musu k(h)၁a \\
`dress on lady'

\ex. k(h) k(h)၁a \\
`dress on man'

\ex. musu \\
`lady'

\ex. kai (long)(low falling)\\
`man' or `rashes'

\ex. kai (short) \\
`son of a woman, name'

\ex. k၁a \\
`monkey'

\ex. kca  c(h)၁ŋ bwæ\\
`caller note'

\ex. d(h)æŋe or d(h)uŋe  \\
`weaving'

\exg. a k၁at͡ʃi \\
3SG wear\\
`He/She's wearing'

\ex. aŋʊ k၁at͡ʃi \\
3-PL wear\\
`They're wearing'

\exg. ŋɐɐ k၁at͡ʃi\\
1EXCL-POSS wear\\
`We're wearing'

\exg. ŋ-ä k၁at͡ʃi\\
1SG-POSS wear\\
`I'm wearing'

\ex. d၁mɐ
`shirt'

\ex. d၁mɐ ʎɛwa
`red shirt'

\ex. d၁mɐ k(h)uŋu
`long-sleeved shirt'

\ex. kuŋdu
`short'

\ex. d၁mɐ soŋhwe
`short-sleeved shirt'

\ex. ʎɛnsäma
`tall/long people'

\ex. bʊŋ(w)faa
`pants'

\ex. bʊŋ(w)fa
`disrespect'

\ex. bʊŋ(w)faa uma
`shorts'

\ex. bʊŋ(w)faa soŋhwe
`pants (long)'


\section{Body Parts - Joey} 



%\jal{I started putting comments but overall you need to go through and reorder all of these -- start with major, visible body parts, and think about what might be useful in future sentence creation; over all we need to move from simple to complex, and you won't get through your list so important to prioritize}

\exg. bóː\\
`Body' \jf{compare with bôː ('arm')}\\

\exg. kuné\\
`Head'\\

\exg. ejă\\
`Eye'
\jf{compare with ejaː ('peanut')}\\

\exg. toː\\
`Ear' \jf{also means 'jealousy'}\\

\exg. súnè\\
`Nose' \jf{also means 'fasting'}\\

\exg. dá̆\\
`Mouth' \jf{consultant notes that this is related to the word for 'food'}\\

\exg. bôː\\
`Arm' \jf{compare with bóː ('body')}\\

\exg. ibôː\\
`Your arm' \\

\exg. mbôː\\
`My arm' \\

\jf{Nasal consonant that has same place of articulation as succeeding consonant appears to be a first person possessive prefix (?)}

\exg. t͡ʃɪ̀ˈnɛ̀̆\\
`Leg' \jf{compare with ˈt͡ʃɪnɛ ('house')}\\
\wml{I had \textipa{t\textesh\'{e}n\`{e}} for house and \textipa{t\textesh\`{e}n\'{e}} for leg}

\exg. kundi\\
`Hair' (head)\\

\exg. buà̆\\
`Beard' \jf{compare with búːwà ('big belly')}\\

\exg. bù̆\\
`Stomach' \jf{compare with búːwà ('big belly')}\\

\exg. búːwà\\
`Big belly' \jf{compare with bù̆ ('stomach') and buwà̆  ('beard')}\\

\exg. ìː\\
`Brain' \jf{also means 'water'}\\



 \section{Flora \& Fauna - Giang} 
\g{I'll probably accompany everything here with photos.}\\
\jal{good! do avoid duplication with food and cooking, and include possible translations when we have them}

Fruits:
\exg. kóŋbwè\\
fruit\\
`Fruit'
\\\g{Anthony noted that it's related to [kónè], `tree'.}
\\\wml{Is it \textipa{\ng\texttoptiebar{gb}}?}

\exg. màá\\
orange\\
`Banana'

\exg. kòŋbíí\\
papaya\\
`Papaya'

\exg. máŋɡò\\
mango\\
`Mango'

\exg. fétù\\
pineapple\\
`Pineapple'\\
\wml{I had [\textipa{f\'{\textepsilon}t\textsuperscript h\^{u}}]}

\ex. `Watermelon'
%he said they don't generally have this kind of fruit

\exg. ɡwèvá\\
guava\\
`Guava'

Q: Is there a fruit of Sierra Leone you like that you want to share with me?

Animals:
\exg. swéé\\
animal\\
`Animal'


\exg. skwà\\
monkey\\
`Monkey'\\
\wml{\textipa{kwa}?}

\ex. `Hippopotamus'

\exg. kàà\\
snake\\
`Snake'

\exg. tʰwá\\
rat\\
`Rat'

\exg. úú\\
dog\\
`Dog'\\
\g{I cannot tell whether it's low or high tone.}

\exg. ɲàŋɡùmá\\
cat\\
`Cat'

Q: Did you have a pet growing up? What is it?

Flowers: \g{Anthony said they generally refer to leaves rather than flowers. It's not an environment where people give out flowers.}
\ex. `Flower'


Misc:
\exg. kòné\\
tree\\
`Tree'

\exg. bándà\\
cotton\_tree\\
`Cotton tree'

\exg. ɡ͜bɛ́ɛ́\\
oil\_palm\\
`Oil palm' and also the wine that comes from the palm trees.
%these are some common species of trees in Sierra Leone

\section{Geography, Meteorology - Mingyang} 
    \ex. ji:\\
        `water'
        
    \ex. ji: ndɛ\\
        small water bodies (`small river', etc.)

    \exg. ji: dɔ:ma\\
        water small\\
        small water bodies
    
    \exg. ji: wa\\
        water big\\
        big water bodies (`big river', `sea', `ocean', etc.)
        
    \ex. kong$^w$ɛ\\
        `mountain'
           
    \exg. f$^j$a wa\\
        bush big\\
        `forest'

    \ex. sona\\
        `rain'

    \ex. sonɛna\\
        `rainy'
        
    \ex. banda\\
        `cloud', `sky'

    \exg. banda finɛ\\
        cloud black\\
        `cloudy'

    \ex. winɛ\\
        `hot'
        
    \ex. t͡ʃi:ma\\
        `cold'

    \ex. t$^h$e:\\
        `sun', `day'
        
    \exg. t$^h$e: dondo\\
        day one\\
        `one day'

    \exg. t$^h$e: f$^j$a\\
        day two\\
        `two days'

    \exg. t$^h$e: sawa\\
        day three\\
        `three days'

    \ex. bi\\
        `today'

    \ex. síná\\
        `tomorrow'

    \ex. kúnú\\
        `yesterday'

    %\mb{Tones in Kono are difficult for me to perceive, but for `tomorrow' and `yesterday', I think they both have HH.}

    \ex. t$^h$e:ma\\
        `dry season'

    \ex. sama\\
        `rainy season'

               
\section{Food, Cooking - Jan} 
\jal{looks like you've looked at some sources, which is great; include possible translations when we have them; also a good chance to ask him what they eat and how they cook it; food \& cooking will be very useful terminology for examples later }
\exg. \textipa{daofine} \\
gloss \\ 
`item meant to be eaten'

\exg. \textipa{fine} \\
gloss \\
`thing'\\
\wml{I think I had \textipa{f\'{e}n\`{e}} for `thing' whereas \textipa{f\'{\i}n\`{e}} was `black'.}
\\
\mb{I hear finɛ for `black' and fenɛ for `thing'.}

\exg. \textipa{du\texttoptiebar{mb}ii} \\ 
gloss \\
`orange' (fruit) 

\exg. \textipa{k\super wee} \\ 
gloss \\
`rice'

\exg. \textipa{ta\ng \'a} \\ 
gloss \\
`cassava'

\exg. \textipa{t\super ha\ng aumba} \\ 
gloss \\
`cassava leaves'

\exg. \textipa{\texttoptiebar{mb}und\'e} \\ 
gloss \\
`potato'

\exg. \textipa{jo\texttoptiebar{gb}\'a} \\ 
gloss \\
`potato leaves'

\exg. \textipa{t\super hueawa} \\ 
gloss \\
`red oil'

\exg. \textipa{jawa} \\ 
gloss \\
`red'

\exg. \textipa{k\super wie} \\ 
gloss \\
`salt'

\exg. \textipa{ej\'a\'a} \\ 
gloss \\
`peanut butter'

\exg. \textipa{ij} \\ 
gloss \\
`water/brain/intelligent'

\exg. \textipa{eab\'asi} \\ 
gloss \\
`onion'

\exg. \textipa{swei} \\ 
gloss \\
`meat'

\exg. \textipa{\texttoptiebar{pf}ut\super hu} \\ 
gloss \\
`pepper'

\exg. \textipa{\texttoptiebar{pf}uu} \\ 
gloss \\
`overseas, abroad'

\exg. \textipa{t\super hee} \\ 
gloss \\
`chicken'

\exg. \textipa{\textltailn ee} \\ 
gloss \\
`fish'

\exg. \textipa{k\super wi} \\
gloss \\
`leopard'

\section{Home - Wesley} 

\exg.  \textipa{t\textesh\'{e}n\`{e}}\\
`house', `home'\\
\wml{Not sure about the tone on the second syllable.}

\exg. \textipa{t\textesh\'{e}m-b\'{a}}\\
house-big(?)\\
`apartment', `big house'\\
\wml{Anthony said you would modify it this way if the house is ``big'', and that it can be used for ``apartment''. He also mentioned that it is not *t\textesh\'{e}nw\'{a}, which suggests there is some kind of allomorphy between -mba and -wa.}

\exg. \textipa{\textturnr\'{u}mu}\\
bedroom\\
`bedroom'\\
\wml{Anthony supplied both \textipa{\textturnr\'{u}mi} and \textipa{\textturnr\'{u}mu}; identified it as an English borrowing.}

\exg. \textipa{s\'{o}fa} \\
bed \\
`bed'

\exg. \textipa{k\'{\i}t\textesh i} \\
kitchen \\
`kitchen'\\
\wml{Generally a separate structure from the house, like toilets.}

\exg. \textipa{p\'{a}lwe} \\
common space in home \\
\wml{Within the household, this is a common open space that leads to all the rooms. It is not exactly a dining space, but where people would spend time talking and eating.}

\exg. \textipa{p\'\textepsilon sa} \\
sitting area outside house \\
\wml{Anthony described this as an exterior sitting area; he likened it to a balcony.}

\exg. \textipa{t\textsuperscript h\'{e}bu} \\
table \\
`table'

\exg. \textipa{s\`{i}} \textipa{f\'{e}n\`{e}} \\
sit thing \\
`chair'

\exg. \textipa{\texttoptiebar{gb}\v{o}} \textipa{s\'{a}} \textipa{f\'{e}n\`{e}}\\
book put thing \\
`bookshelf'

\exg. \textipa{b\'{o}}\\
padlock\\
`padlock'

\exg. \textipa{b\^{o}}\\
curse\\
`to curse somebody'

\exg. \textipa{l\'{a}mpi}\\
lamp \\
`lamp'

\exg. \textipa{m\'{a}m} \textipa{f\'{e}n\`{e}}\\
light thing \\
`lamp'\\
\wml{`To light' is \textipa{m\'{a}ne}. In the compound, I can't tell what happens to the nasal --- it sounds like it's bilabial, or maybe labiodental due to assimilation.}

\section{Occupations} 


\section{Leisure - Alex C} 

\jal{Glad you're looking at the grammar, but also this will be a good chance to ask him what they do for leisure - games, sports, etc, and get words for that. Do also use google for ideas -- e.g. a quick search says that soccer is a major sport there}

\jal{note that it can be tricky to elicit a verb in isolation; not very natural for speakers}

\jal{quotations in latex need to use these symbols:  `xyz'}

\exg.  tácɛ̂\\
gloss \\
`(to) walk (formal)` \\
Foday-Ngongou: tácɛ̂ (verb) tácê (noun)

\exg. bembɛnɛ̂\\
gloss\\ 
`to walk (informal)`

\exg. sâ\super{m}b\super{h}a\\
specific type of drum or all drums? \\
`drum`

\exg. example\\
gloss\\
`pleasure, enjoyment` \\
Foday-Ngongou: jɔ́ù

\exg. example\\
gloss\\
`(to) write` \\
Foday-Ngongou: ɲɛ̀ɲ

\exg. example\\
gloss\\
`(to) laugh`
Foday-Ngongou \\ jɛ́ŋ

\exg. ˈfʊtbɔːli\\
specific type of drum or all drums? \\
`football (central kono)`

\exg. T͡ʃ\super{j}ɛnɖ͡ʐɛnɛ\\
specific type of drum or all drums? \\
`football (sando(?) kono)`


\exg. tɒmbwɛ\\
a specific type of dance or all dances? \\
`dance`

\exg. bód\super{ɪ}ja\\
related to the word for stomach/belly \\
`fun, happpines`

\exg. s\super{u}winɛ\\
gloss\\ 
`rest`


\exg. si\\
gloss\\ 
`music`


\exg. ɲɛp\super{h}ɑɪ́nɛ̂\\
ɲɛ – fish, p\super{h}ɑɪ́nɛ̂ – to chase something \\
`to fish`

\exg. kanɛ̂\\
do all verbs end in ɛ̂? \\
`to read, to learn`



\newpage 

\end{document}
